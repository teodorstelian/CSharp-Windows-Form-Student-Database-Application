\documentclass{beamer}

\title{Industrial Informatics Semester Project}
\subtitle{Technical University of Cluj-Napoca}

\usepackage{graphicx}
\graphicspath{ {./images/} }

\usepackage[rightcaption]{sidecap}

\usepackage{wrapfig}

\setbeamertemplate{caption}[numbered]

\author[Technical University of Cluj-Napoca]{%
	\texorpdfstring{%
		\begin{columns}
			\column{.5\linewidth}
			\centering
			Achim Daniel \\ Băltărețu Teodor-Stelian \\ Bugnariu Vlad \\ Ciobotaru Alexandru
			\column{.5\linewidth}
			\centering
			Fildan Claudiu \\ Furdui Vasile Teodor \\ Gog Ionela-Maria
		\end{columns}
	}
	{Author 1, Author 2, Author 3}
}

\mode<presentation>
\usetheme{Darmstadt}

\begin{document}
	\maketitle
	
    \begin{frame}
    	\frametitle{Table of Contents}
    	\tableofcontents
    \end{frame}	

\AtBeginSection[]
{
	\begin{frame}
		\frametitle{Table of Contents}
		\tableofcontents[currentsection]
	\end{frame}
}

%   PRESENTATION SLIDES

%------------------------------------------------
\section{Overview}
\begin{frame}{Purpose}
	
	\begin{itemize}
		\item Guide High School students to their future career based on their skills and personality.
	\end{itemize}

	\begin{figure}[t]
		\includegraphics[height=6 cm]{apzero}
		\centering
	\end{figure}
\end{frame}

%------------------------------------------------
\section{User Story}
\begin{frame}{User Story}
	
	\begin{block}{General User Story}
		As a high school student I want to find a way to choose my future career
		so that I can start to prepare for it.
	\end{block}
	
\end{frame}

%------------------------------------------------

\begin{frame}{User Story}
	
	\begin{itemize}
		\item As a high-school student I want to
		have an account so that I can review the 
		information later.
	\end{itemize}
	
	\begin{itemize}
		\item As a high-school student I want to
		discover which academic field is suitable to me so that I can choose a career I would like.
	\end{itemize}
	
	\begin{itemize}
		\item As a high-school student I want to find universities which offers the study program suitable for my career path so that I can narrow my search.
	\end{itemize}
	
\end{frame}

%------------------------------------------------

\begin{frame}{User Story}
	
	\begin{itemize}
		\item As a high-school student I want to
		find information about the academic environment so that I will have a better perspective of my life there.
	\end{itemize}
	
	\begin{itemize}
		\item As a high-school student I want to
		find an estimation about the cost of life
		so that I can know if I can afford to attend a certain university.
	\end{itemize}
	
	\begin{itemize}
		\item As a high-school student I want to find information about the future perspective of a career so that I can follow a job which will be in high demand in the future.
	\end{itemize}
	
\end{frame}
%------------------------------------------------
\section{Survey}

\AtBeginSection[]
{
	\begin{frame}
		\frametitle{Table of Contents}
		\tableofcontents[currentsection]
	\end{frame}
}

%------------------------------------------------

\begin{frame}{Survey}
	
	\begin{itemize}
		\item On a survey, 60 former and current High-School students were asked a set of questions regarding their future plan.
	\end{itemize}
	
	\begin{figure}[t]
		\includegraphics[height=4.5cm]{pone}
		\centering
		\caption{First question}
	\end{figure}
\end{frame}

%------------------------------------------------

\begin{frame}{Question no. 2}
	
	\begin{itemize}
		\item Figure 2 demonstrates the lack of cooperation between students and the career mentors.
	\end{itemize}
	
	\begin{figure}[t]
		\includegraphics[height=4.5cm]{ptwo}
		\centering
		\caption{Second question}
	\end{figure}
\end{frame}

%------------------------------------------------

\begin{frame}{Question no. 3}
	
	\begin{itemize}
		\item Figure 3 shows that one third of the students haven't decided about their future yet.
	\end{itemize}
	
	\begin{figure}[t]
		\includegraphics[height=4.5cm]{pthree}
		\centering
		\caption{Third question}
	\end{figure}
\end{frame}

%------------------------------------------------

\begin{frame}{Question no. 4}
	
	\begin{itemize}
		\item Figure 4 displays some of the toughest issues students encountered when it comes to their future.
	\end{itemize}
	
	\begin{figure}[t]
		\includegraphics[height=4.5cm]{pfour}
		\centering
		\caption{Fourth question}
	\end{figure}
\end{frame}

%------------------------------------------------

\begin{frame}{Question no. 5}
	
	\begin{itemize}
		\item Figure 5 illustrates that students have the old mentality in which a faculty is choose based on the knowledge, not based on the interest or inclination to the subject.
	\end{itemize}
	
	\begin{figure}[t]
		\includegraphics[height=4.5cm]{pfive}
		\centering
		\caption{Fifth question}
	\end{figure}
\end{frame}

%------------------------------------------------

\begin{frame}{Question no. 6}
	
	\begin{itemize}
		\item In Figure 6 it is shown that three quarters of students manifest interest in such an application.
	\end{itemize}
	
	\begin{figure}[t]
		\includegraphics[height=4.5cm]{psix}
		\centering
		\caption{Sixth question}
	\end{figure}
\end{frame}

%------------------------------------------------

\begin{frame}{Question no. 7}
	
	\begin{itemize}
		\item Based on this question we will determine the features of this application
	\end{itemize}
	
	\begin{figure}[t]
		\includegraphics[height=4.5cm]{pseven}
		\centering
		\caption{Seventh question}
	\end{figure}
\end{frame}

%------------------------------------------------

\section{Goals}
\begin{frame}{Goals}
	
	\begin{itemize}
		\item Based on the results of the survey, our goal is to create an application that can guide the students to a better choice.
	\end{itemize}

	\begin{figure}[t]
		\includegraphics[height=4.5cm]{aptwo}
		\centering
	\end{figure}
	
\end{frame}

%------------------------------------------------

\section{Benefits}
\begin{frame}{Benefits}
	
	\begin{itemize}
		\item This will help them to save time lost from choosing a wrong domain and to find their unlocked potential.
	\end{itemize}

	\begin{itemize}
		\item The application will choose the appropriate universities by using the user's skills and personality.
	\end{itemize}

	\begin{itemize}
		\item By choosing the right domain, the society evolves faster with a smaller percentage of unemployment, higher level of employee productivity and better use of the resources used in the educational system.
	\end{itemize}

\end{frame}

%------------------------------------------------

\begin{frame}{Thank you for your attention!}
	\begin{figure}[t]
		\includegraphics[height=4.5cm]{apthree}
		\centering
	
	\end{figure}
	
	
\end{frame} 

%------------------------------------------------

\end{document} 
    
	
